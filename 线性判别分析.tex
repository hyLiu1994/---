% Created 2020-04-08 Wed 21:14
% Intended LaTeX compiler: pdflatex
\documentclass[11pt]{article}
\usepackage[utf8]{inputenc}
\usepackage[T1]{fontenc}
\usepackage{graphicx}
\usepackage{grffile}
\usepackage{longtable}
\usepackage{wrapfig}
\usepackage{rotating}
\usepackage[normalem]{ulem}
\usepackage{amsmath}
\usepackage{textcomp}
\usepackage{amssymb}
\usepackage{capt-of}
\usepackage{hyperref}
\usepackage{ctex}
\author{hyliu}
\date{\today}
\title{}
\hypersetup{
 pdfauthor={hyliu},
 pdftitle={},
 pdfkeywords={},
 pdfsubject={},
 pdfcreator={Emacs 26.3 (Org mode 9.2.6)}, 
 pdflang={English}}
\begin{document}

\tableofcontents

\section{线性判别分析}
\label{sec:org65a9e05}
\subsection{问题定义}
\label{sec:orgcad4e4f}
数据数学表示形式为:
\begin{equation}
\label{eq:1}
\begin{align}
&X = \left( x_1, x_2,...,x_{N} \right)^T = \left (
\begin{array}{c}
x_{1}^T \\
x_2^T \\
\vdots \\
x_N^T 
\end{array}
\right )_{N*P}\\
&Y=\left(\begin{array}{l}y_{1} \\ y_{2} \\ \vdots \\ y_{N}\end{array}\right)_{N \times 1}\\
\end{align}
\end{equation}
\begin{equation}
\label{eq:3}
\begin{align}
&\left\{\left(x_{i}, y_{i}\right)\right\}_{i=1}^{N}, x_{i} \in \mathbb{R}^{p}, y_{i} \in\{+1,-1\}\\
&x_{c_1}=\left\{x_{i} | y_{i}=+1\right\}, \quad x_{c_{2}}=\left\{x_{i} | y_{i}=-1\right\}\\
&\left| x_{c_1} \right| = N_1, \left| x_{c_2} \right| = N_2, N_1 + N_2 = N
\end{align}
\end{equation}
其中 \(c_1 = +1, c_2 = -1\), \(N, N_1, N_2\) 分别表示样本总数,正样本数量以及负样本数量, \(x_{c_1}, x_{c_2}\) 分布表示正样本集合与负样本集合。
\subsection{核心思想}
\label{sec:org0419d5a}
类内小,类间大 (高内聚,低耦合)
\subsection{目标函数}
\label{sec:orgc4ec09c}
\subsubsection{目标函数的设计}
\label{sec:org6cda1d2}
为了实现 "类内小,类间大" 的目标。
我们要使得,不同类别的均值之间的差值尽可能大,各个类别类内的方差尽可能小。
单一类别的均值与方差可以表示为如下形式。
\begin{equation}
\label{eq:5}
$\begin{aligned} z_{i} &=w^{\top} x_{i} \\ \bar{z} &=\frac{1}{N} \sum_{i=1}^{N} z_{i}=\frac{1}{N} \sum_{i=1}^{N} w^{\top} x_{i} \\ S_{z} &=\frac{1}{N} \sum_{i=1}^{N}\left(z_{i}-\bar{z}\right)\left(z_{i}-\bar{z}\right)^{\top} \\ &=\frac{1}{N} \sum_{i=1}^{N}\left(w^{\top} x_{i}-\bar{z}\right)\left(w^{\top} x_{i}-\bar{z}\right)^{\top} \end{aligned}$
\end{equation}
因此更进一步,\(c_1, c_2\) 类别对应的均值与方差可以表示为如下形式:
\begin{equation}
\label{eq:6}
\begin{aligned}
&C_{1}: \bar{z}_{1}=\frac{1}{N_{1}} \sum_{i=1}^{N_{1}} w^{\top} x_{i}\\
&S_{1}=\frac{1}{N_{1}} \sum_{i=1}^{N_{1}}\left(w^{\top} x_{i}-\bar{z}_{1}\right)\left(w^{\top} x_{i}-\bar{z}_{1}\right)^{\top}\\
&C_{2}: \bar{z}_{2}=\frac{1}{N_{2}} \sum_{i=1}^{N_{2}} w^{\top} x_{i}\\
&S_{2}=\frac{1}{N_{2}} \sum_{i=1}^{N_{2}}\left(w^{T} x_{i}-\bar{z}_{2}\right)\left(w^{\top} x_{i}-\bar{z}_{2}\right)^{\top}
\end{aligned}
\end{equation}
类内距离可以表示为:\(S_1 + S_2\)
类间距离可以表示为: \(\left( \bar{z}_1 - \bar{z}_{2} \right)^{2}\)
要达到类内小,类间大的效果, 目标函数最终设计为:
\begin{equation}
\label{eq:7}
J \left( w \right) = \frac{ \left( \bar{z}_1 - \bar{z}_2 \right)^2}{S_1 + S_2}
\end{equation}
\subsubsection{目标函数的推导}
\label{sec:orgc8a40ea}
对目标函数 \(J \left( w \right) = \frac{ \left( \bar{z}_1 - \bar{z}_2 \right)^2}{S_1 + S_2}\) 分子与分母进行进一步的展开。
           \begin{equation}
\label{eq:8}
\begin{aligned}
molecule &=\left(\frac{1}{N_{1}} \sum_{i=1}^{N_1} w^{\top} x_{i}-\frac{1}{N_{2}} \sum_{j=1}^{N_{2}} w^{\top} x_{i}\right)^2=\left[w^{\top}\left(\frac{1}{N_{1}} \sum_{i=1}^{n} x_{i}-\frac{1}{N_{2}} \sum_{i=1}^{n} x_{i}\right)\right]^2\\
&=\left(w^{\top}\left(\bar{x}_{c_1}-\bar{x}_{c_{2}}\right)\right)^{2}=w^{\top}\left(\bar{x}_{c_1}-\bar{x}_{c_{2}}\right)\left(\bar{x}_{c_{1}}-\bar{x}_{c_{2}}\right)^{\top} \cdot w
\end{aligned}
\end{equation}
\begin{equation}
\label{eq:9}
\begin{aligned}
S_{1} &=\frac{1}{N_{1}} \sum_{i=1}^{N_{1}}\left(w^{\top} x_{i}-\frac{1}{N_{1}} \sum_{j=1}^{N} w^{\top} x_{j}\right)\left(w^{\top} x_{i}-\frac{1}{N_{1}} \sum_{j=1}^{N_{1}} w^{\top} x_{j}\right)^{\top} \\
&=\frac{1}{N_{1}} \sum_{i=1}^{N_{1}} w^{\top}\left(x_{i}-\quad \bar{x}_{c_{1}}\right)\left(x_{i}-\bar{x}_{c_1}\right)^{\top} w \\
&=w^{\top}\left[\frac{1}{N_{1}} \sum_{i=1}^{N}\left(x_{i}-\overline{x_{c_1}}\right)\left(x_{i}-\bar{x}_{c_1}\right)^{\top}\right] w \\
&=w^{T} \cdot S_{c_{1}} \cdot w \\
&=w^{\top} S_{c_{1}} w
\end{aligned}
\end{equation}
\begin{equation}
\label{eq:10}
\begin{align}
Denominator &= S_1 + S_2\\ 
&= w^T S_{c_1} w + w^T S_{c_2} w \\
&= W^T \left( S_{c_1} + S_{c_2} \right) w
\end{align}
\end{equation}
所以最终目标函数可以表示为
\begin{equation}
\label{eq:12}
J \left( w \right) = \frac{w^T \left( \bar{x}_{c_1} - \bar{x}_{c_2} \right)\left( \bar{x}_{c_1} - \bar{x}_{c_2} \right)^T w}{w^T \left( S_{c_1} + S_{c_2} \right) w}
\end{equation}
\subsection{参数求解}
\label{sec:org68e3a7c}
根据\hyperref[sec:orgc8a40ea]{上一节}我们推导出目标函数为如下形式 
\begin{equation}
J \left( w \right) = \frac{w^T \left( \bar{x}_{c_1} - \bar{x}_{c_2} \right)\left( \bar{x}_{c_1} - \bar{x}_{c_2} \right)^T w}{w^T \left( S_{c_1} + S_{c_2} \right) w}
\end{equation}
在此我们将会推导 \(w\) 。 
首先我们为了简化运算使得
\begin{equation}
\label{eq:13}
\begin{aligned}
&S_{b}=\left(\bar{x}_{c_1}-\bar{x}_{c_{2}}\right)\left(\bar{x}_{c_1}-\bar{x}_{c_2}\right)^{T}\\
&S_{w}=S_{c_1}+S_{c_{2}}\\
&J \left( w \right) = \frac{w^T S_b w}{w^T S_w w} = w^T S_b w (w^T S_w w)^{-1}
\end{aligned}
\end{equation}
进一步,我们使得 \(\frac{\partial J \left( w \right)}{\delta w} = 0\) 以推导 \(w\) (这里假设 \(w\) 的最优解为导数为0的时候) 。
\begin{equation}
\label{eq:16}
\begin{align}
\label{eq:18}
\frac{\partial J \left( w \right)}{\partial w} = 2 S_b w \cdot & \left( w^T S_w w \right)^{-1} + w^T S_b w \cdot \left( -1 \right) \left( w^T S_w w \right)^{-2} \cdot 2 S_w w  = 0\\
S_b w \left( w^T S_w w \right) &= \left ( w^T S_b w \right ) S_w w, \quad \left( w^T S_w w \right), \left( w^T s_w w \right) \in \mathbb{R}\\
S_w w &= \frac{w^T S_w w}{w^T S_b w} S_b w\\
    w &= \frac{w^T S_w w}{w^T S_b w} S_w^{-1} S_b w \\
    w &\propto S_w^{-1} S_b w\\
    w &\propto S_w^{-1} (\bar{x}_{c_1} - \bar{x}_{c_2}) (\bar{x}_{c_1} - \bar{x}_{c_2})^T w, \quad (\bar{x}_{c_1} - \bar{x}_{c_2})^T w \in \mathbb{R}\\
    w &\propto S_w^{-1} \left( \bar{x}_{c_1} - \bar{x}_{c_2} \right)
\end{align}
\end{equation}
当 \(S_w\) 对角且各项同性, \(S_w^{-1} \propto I\) 
\begin{equation}
\label{eq:19}
w  \propto \left( \bar{x}_{c_1} - \bar{x}_{c_2} \right)
\end{equation}
\end{document}