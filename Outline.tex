% Created 2020-04-06 Mon 21:51
% Intended LaTeX compiler: pdflatex
\documentclass[11pt]{article}
\usepackage[utf8]{inputenc}
\usepackage[T1]{fontenc}
\usepackage{graphicx}
\usepackage{grffile}
\usepackage{longtable}
\usepackage{wrapfig}
\usepackage{rotating}
\usepackage[normalem]{ulem}
\usepackage{amsmath}
\usepackage{textcomp}
\usepackage{amssymb}
\usepackage{capt-of}
\usepackage{hyperref}
\usepackage{ctex}
\author{hyliu}
\date{\today}
\title{}
\hypersetup{
 pdfauthor={hyliu},
 pdftitle={},
 pdfkeywords={},
 pdfsubject={},
 pdfcreator={Emacs 26.3 (Org mode 9.2.6)}, 
 pdflang={English}}
\begin{document}

\tableofcontents

\section{基础}
\label{sec:orga260cf5}
\subsection{\href{频率派VS贝叶斯派.org}{频率派VS贝叶斯派}}
\label{sec:org3491100}
\subsection{数学基础}
\label{sec:orgb166ea1}
\subsubsection{\href{高斯分布.org}{高斯分布}}
\label{sec:org2a5f9b5}
\subsubsection{\href{线性代数内容.org}{线性代数}}
\label{sec:org260a4a1}
\section{线性回归}
\label{sec:orgf822865}
\subsection{\href{线性回归.md}{最小二乘法}}
\label{sec:org5732098}
\subsection{\href{线性回归.md}{最大释然估计}}
\label{sec:orga29245b}
\subsection{\href{线性回归.md}{正则化}}
\label{sec:org0017be6}
\subsubsection{\href{线性回归.md}{岭回归}}
\label{sec:org0f4a600}
\subsubsection{\href{线性回归.md}{概率视角的正则化}}
\label{sec:orgafff404}

\subsection{线性回归的特点}
\label{sec:org9869aeb}
\subsubsection{线性}
\label{sec:org8c580a1}
基于此拓展的模型
\begin{enumerate}
\item 属性非线性: 特征转换 (多项式回归)
\label{sec:orgcf7865b}
\texttt{解释}
\item 全局非线性: 线性分类 (激活函数是非线性)
\label{sec:org80401fc}
\item 系数非线性: 神经网络, \href{感知机.org}{感知机}
\label{sec:orgd568c6b}
随机初始化参数,并且使用迭代算法求解。
\end{enumerate}
\subsubsection{全局性}
\label{sec:orgf4a788b}
基于此拓展的模型
\begin{enumerate}
\item 线性样条回归
\label{sec:org51846b2}
\item 决策树
\label{sec:org184b605}
\end{enumerate}
\subsubsection{数据未加工}
\label{sec:orga995de2}
基于此拓展的模型
\begin{enumerate}
\item PCA
\label{sec:org6a9f713}
\item 流型
\label{sec:org856a12b}
\end{enumerate}
\section{线性分类}
\label{sec:orgc8dd216}
线性分类是基于\hyperref[sec:orgf822865]{线性回归}的进一步拓展。
\subsection{线性分类的类别}
\label{sec:orgdb341de}
\subsubsection{硬分类}
\label{sec:org3863831}
\(y \in \left\{ 0,1 \right\}\)
\begin{enumerate}
\item \href{线性判别分析.org}{线性判别分析}
\label{sec:org9e4236e}
\item \href{感知机.org}{感知机}
\label{sec:org38528ff}
\end{enumerate}
\subsubsection{软分类}
\label{sec:orgf7d7146}
\(y \in \left[ 0,1  \right]\)
\begin{enumerate}
\item 生成式: Gaussian Discriminant Analysis
\label{sec:org8d2f5d5}
\item 判别式: Logistic Regression
\label{sec:org501a54e}
\end{enumerate}
\subsection{线性分类中的非线性}
\label{sec:org9243dd5}
线性分类中的非线性主要来自两个方面:
\begin{enumerate}
\item 激活函数的非线性
\item 降维
\end{enumerate}

\section{经典算法}
\label{sec:orgf2cce2d}
\subsection{线性高斯模型}
\label{sec:org4caddef}
\subsection{K近邻}
\label{sec:org841087f}
\subsection{朴素贝叶斯}
\label{sec:orgd49e1a0}
\subsection{决策树}
\label{sec:orge127ef9}
\subsection{逻辑回归}
\label{sec:org204462d}
\subsection{支持向量机}
\label{sec:orgd0f0ebd}
\subsection{Boosting}
\label{sec:orgbf1eebc}
\subsection{EM算法}
\label{sec:orgdf49d52}
\subsection{隐马尔科夫模型}
\label{sec:org3be65da}
\subsection{条件随机场}
\label{sec:orgae9b8a6}
\subsection{线性模型---回归算法}
\label{sec:org7ea4338}
\subsection{分类算法}
\label{sec:orgf548733}
\subsection{神经网络}
\label{sec:org963cfde}
\subsection{核方法}
\label{sec:orgf097cc7}
\subsection{稀疏核机}
\label{sec:org522c237}
\subsection{概率图模型}
\label{sec:org82a0d7a}
\subsection{混合模型}
\label{sec:orgf23585c}
\subsection{近似算法}
\label{sec:org21f07f2}
\subsection{采样算法}
\label{sec:org0f2a5ae}
\subsection{连续性随机变量}
\label{sec:orge69c7bc}
\subsection{顺序数据}
\label{sec:orgaa35ce9}
\subsection{组合模型}
\label{sec:org3740017}
\section{学习资料}
\label{sec:orgdffa043}
\subsection{书籍}
\label{sec:org2ed51b4}
\subsubsection{统计学习方法 李航}
\label{sec:org89d006d}
\subsubsection{"西瓜书" 周志华 (百科全书)}
\label{sec:orgad70185}
\subsubsection{Pattern Recognition and Machine Learning, PRML}
\label{sec:org082fabb}
\subsubsection{Machine Learning:A Probabilistic Perspective, MLAPP (百科全书)}
\label{sec:org9fef3c6}
\subsubsection{The Elements of Statistical Learning, ESL}
\label{sec:org4c8e872}
\subsubsection{Deep Learning (DL)}
\label{sec:org364a2a3}
\subsection{视频}
\label{sec:orga2648fe}
\subsubsection{台大 林轩田}
\label{sec:orgafe4052}
\begin{enumerate}
\item 机器学习基石 (VC Theory, 正则化, 线性模型)
\label{sec:orgab1316a}
\item 机器学习技法 (SVM, 决策树, 随机森林, 神经网络, Deep Learning)
\label{sec:org8f5edbd}
\end{enumerate}
\subsubsection{张志华}
\label{sec:org85edb7d}
\begin{enumerate}
\item 机器学习导论 (频率派)
\label{sec:org7d2c0b0}
\item 统计机器学习 (共轭理论, 贝叶斯派, 偏数学)
\label{sec:orga5992b0}
\end{enumerate}
\subsubsection{Ng, 吴恩达}
\label{sec:org1c46dd6}
\begin{enumerate}
\item CS229
\label{sec:org409c8cf}
\end{enumerate}
\subsubsection{徐亦达}
\label{sec:org6e6ade7}
\begin{enumerate}
\item 概率模型 (EM, HMM)
\label{sec:org97e26fd}
\item github -> notes
\label{sec:org7c1694e}
\end{enumerate}
\subsubsection{台大 李宏毅}
\label{sec:org6671a39}
\begin{enumerate}
\item ML 2017 (Deep Learning)
\label{sec:orgfdbecac}
\item MLDS 2018 (Deep Learning)
\label{sec:org51aebab}
\end{enumerate}
\end{document}